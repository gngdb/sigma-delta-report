\section{Verification}
\label{Verification}
%section including all the simulation results created

    \subsection{DC Stability}
    \label{Verification:DCstab}
    %tests showing DC stability with different problems
    Tones at the output of a modulator will degrade the SQNR performance of a circuit during normal operation.
    Also, the effects of DC inputs on the stability of the loop must also be investigated to find the input range of the modulator over which it is stable. 
    The effect on output DC response from an input DC sweep due to finite amplifier gain is another important factor.

    Shown in figure \ref{fig:tranexample} is the transient output at various points in the modulator.
    It can be seen to be stable for the DC input voltage being applied at %some number of V

    %should have a plot of just the transient voltages at different points in the circuit for high input voltages to show nothing is becoming linear
    %GET RID OF THIS IF ITS DIFFICULT TO SOURCE THE WAVE
    \begin{figure}
        \begin{center}
        \includegraphics[width=0.8\textwidth]{tranexample}
        \label{fig:tranexample}
        \caption{An example transient output waveform.}
        \end{center}
    \end{figure}


        \subsubsection{DC input voltage - DC output voltage}
        \label{Verification:DCinputoutput}
        %point out how this design's dead zone performance is better than for other designs
        %so it can be implemented using amplifiers with lower gain
        The effect of finite gain amplifiers on a design can be to cause ``dead zones'' to appear in the transfer function of DC input to DC output.
        This design is expected to be resistant to this effect as it is effectively a third order design and the width of the dead zone can be described: 

        %equations describing the dead zone of a third order modulator
        \begin{align}
            
        \end{align}

        %there is no spec for this so could just drive it down very low
        The results of this design to this test are shown in figure \ref{fig:dcdc}.
        No dead zones are visible at any zoom level - the dead zones produced are smaller than the resolution of the sweep.

        \begin{figure}
            \begin{center}
            \includegraphics[width=0.8\textwidth]{dcdc.png}
            \label{fig:dcdc}
            \caption{DC input voltage to DC output voltage for a finite gain of ...} %FILL IN THE GAIN HERE
            \end{center}
        \end{figure}       

        \subsection{DC input voltage - baseband power}
        \label{Verification:DCbbpwr}
        %
        A sweep of DC input voltages while monitoring the baseband power will detect tones appearing at those input voltages.
        The result of this test is shown in figure \ref{fig:dcdcbbpwr}.
        This was run with saturation limits in both integrators equal to 2.546V, to reflect the signal range required of the real circuits.

         \begin{figure}
            \begin{center}
            \includegraphics[width=0.8\textwidth]{dcdcbbpwr}
            \label{fig:dcdcbbpwr}
            \caption{Nominal baseband power resulting from a sweep of DC inputs.}
            \end{center}
        \end{figure}       
        
        %this might result in requiring dither
        The baseband power never exceeds -103dB, so it can be expected that the circuit is well within specification.

        %investigate the effect of mismatch?

%    \subsection{Flat Noise Profile}
%    %using dither to make the noise profile flat
%    The noise profile of the chosen design without dither can be seen in figure \ref{fig:MASH}.
%    It does not meet the specifications in that it is not flat.
%    This is a problem that can be solved by applying dither to the circuit in the form of adding random noise at 
%
%    \begin{figure}
%        \begin{center}
%        \includegraphics[width=0.8\textwidth]{SQNRdither}
%        \label{fig:SQNRdither}
%        \caption{Single simulation showing the effect of dither on the noise floor of the modulator.}
%        \end{center}
%    \end{figure}  
%
%    Dither is therefore applied in all SQNR simulations involving this modulator.

    \subsection{SQNR Stability}
    %sweeps of SQNR
    To verify the performance of a Sigma-Delta modulator it's necessary to verify the resolution of the circuit at many input amplitudes and in cases involving non-ideal components.
    The cases tested here are:
    \begin{itemize}
        \item Capacitor mismatch
        \item Finite gain
        \item Stage mismatch
        \item Finite signal swing
    \end{itemize}

    Without mismatch, but with a finite gain of 1000 and a finite signal swing equal to that expected in the amplifier of 2.546V differential, the results of a sweep of input sine amplitudes is shown in figure \ref{fig:SQNRnominal}.

    \begin{figure}
        \begin{center}
        \includegraphics[width=0.8\textwidth]{SQNRnominal}
        \label{fig:SQNRnominal}
        \caption{SQNR resulting from a sweep of different input amplitudes of a sine wave without mismatch is shown.}
        \end{center}
    \end{figure}  



        \subsubsection{Capacitor Mismatch}
        %capacitor mismatch, the effect of
        Capacitor mismatch causes a mismatch in the multiplication stages of the amplifier.
        In the matlab model there are various precise multiplication elements that can be varied.
        The variation expected in capacitors is 0.4\%, so a variation in these values of 0.4\% must be tested.
 
        \begin{figure}
            \begin{center}
            \includegraphics[width=0.8\textwidth]{SQNRnominal}
            \label{fig:SQNRcap}
            \caption{SQNR resulting from a sweep of different input amplitudes of a sine wave with capacitor mismatch is shown.}
            \end{center}
        \end{figure}             

        \subsubsection{Effect of Transfer Function Mismatch} 
        %run SQNR sweeps with mismatches in the transfer function
        %reference equations describing this from the textbook
        

    \subsection{Cadence Tests}
    %test showing that the sizing of switches and caps is ok

\subsection{Conclusion}
\label{Verification:conclusion}
%conclusion to the verification section

